%!TEX program = xelatex 
\documentclass[a4paper,10pt]{article}
\usepackage[utf8]{inputenc}
\usepackage{amssymb,amsmath,amsfonts}
\usepackage{hyperref}
\usepackage[a4paper, margin=.5in]{geometry}

% 添加中文编译支持(需要使用xelatex编译)
\usepackage{xeCJK}

%opening
\title{文档:听觉皮层的激活-抑制交互统一机理模型}
% \author{Youngmin Park}

\newcommand{\y}[1]{{\texttt{#1}}}

\begin{document}

\maketitle

%\begin{abstract}

%\end{abstract}

\section{Introduction}
The primary tools of this paper are Python (with numpy, scipy, matplotlib), XPPAUTO (XPPAUT or XPP for short) \cite{ermentrout2002simulating}, a Python binding for XPP, \y{Py\_XPPCALL} \cite{xppy}, and Brian2 \cite{Stimberg595710}. Please contact me with any questions about these programs: \y{ympark1988@gmail.com}\footnote{I used XPP extensively in my PhD, and XPP was created by and continues to be developed by Bard Ermentrout, my doctoral advisor. In addition, I am one of the primary contributors to \y{Py\_XPPCALL}.}.

I highly recommend using Ubuntu, since the most stable versions of XPP are currently on Ubuntu. At the time of writing, I have used Ubuntu 16.04 (Python 2.7.12, numpy 1.15.2, scipy 1.1.0, matplotlib 2.2.3), Ubuntu 18.04.2 (Python 2.7.15rc1, numpy 1.15.2, scipy 1.1.0, matplotlib 2.2.3), and XPPAUT version 8 to successfully run all code.

\section{Rate Models}

Rate models are implemented in XPPAUTO. In order to plot the simulations, I run XPPAUTO through the Python binding \y{Py\_XPPCALL}. 

\subsection{XPPAUTO}
All rate models were implemented in \texttt{XPPAUTO} for convenience. XPP auto uses an almost plain-text file format for simulating ordinary differential equations (ODEs). There is extensive documentation on XPPAUTO on Bard's website:
\url{http://www.math.pitt.edu/~bard/xpp/help/xppodes.html}. The linked page provides full documentation of ode file syntax. XPPAUTO also comes with a graphical user interface (GUI), but this will not be needed for this documentation. The XPP .ode files for this project are located in the directory \url{https://github.com/geffenlab/park_geffen/rate_models/xpp}.

Let us take a look at one of the ode files, \texttt{cols3\_fs.ode}, which is the three-unit rate model for the forward suppression paradigm. The file begins with a function definition and a bunch of parameters:
\begin{verbatim}
f(x)=al*heav(x)*x*heav(1/al-x)+heav(x-1/al)
dan(x)=exp(-tl*x)
p al = 3

p wee=1.1,wep=2,wes=1
# either wpe=1.2, or wpp=1.8 to reduce SOM disinhibition activity.
p wpe=1.,wpp=2,wps=2

p wse=6,wss=0,wsp=0
p wee2=1,wpe2=1.25,wse2=.125
#p wee2=1,wpe2=1.5,wse2=.25

par tau1=1,tau2=1
par et=.7,pt=1,st=1
par pv_opto=0,som_opto=0

p q=1.3,tl=1
p taud1=150,taud2=10
p tde1=150,tde2=2
p lat=.65
\end{verbatim}
Comment lines are denoted by the pound symbol (\#). The function \texttt{f} is the firing-rate function, which corresponds to Equation (2) in the manuscript (There is a typo in the current manuscript. All simulations were run using the correct $f$ as shown in the code). The function is zero until $x=0$, at which point the function increases linearly with slope $\text{al}=3$ as a function of $x$, until the function reaches the value 1 (when $x=1/\text{al}$, at which point the function ``saturates'' at the value 1. The second function, which I named \y{dan}, is a decaying exponential with rate \y{tl}, which we will later use to modulate the square-wave inputs into the model. We use the same firing rate function for all models, which is a common choice in modeling studies \cite{natan2015complementary,yarden2017stimulus}.

The notation \texttt{p} and \texttt{par} both denote a line consisting of parameters. The first set of parameters:
\begin{verbatim}
p wee=1.1,wep=2,wes=1
p wpe=1.,wpp=2,wps=2

p wse=6,wss=0,wsp=0
p wee2=1,wpe2=1.25,wse2=.125
\end{verbatim}
are for recurrent excitation (\texttt{wee}), PV-to-Pyr inhibition (\texttt{wep}), SOM-to-Pyr inhibition (\texttt{wes}), Pyr-to-PV excitation (\texttt{wpe}), recurrent PV inhibition (\y{wpp}), SOM-to-PV inhibition (\y{wps}), Pyr-to-SOM excitation (\y{wse}), PV-to-SOM inhibition (\y{wsp}), and recurrent SOM inhibition (\y{wss}). The remaining terms are for the lateral weights. Lateral Pyr-to-Pyr (\y{wee2}), lateral Pyr-to-PV (\y{wpe2}), and lateral Pyr-to-SOM (\y{wse2}). These parameters appear in Equation 4, and in the unit equations on page 2.

The next set of parameters:
\begin{verbatim}
par tau1=1,tau2=1
par et=.7,pt=1,st=1
par pv_opto=0,som_opto=0
\end{verbatim}
Are time constants \y{tau1} and \y{tau2} are for the inhibitory populations, PV, and SOM, respectively. This rate model is written so that 1 time unit is 10ms, i.e., \y{tau1=1} means $\tau_1 = 10$ms \cite{tsodyks1997neural,natan2015complementary}. If needed, one could perform the necessary transformation in time so the model displays the correct time units by default.

The next line is for thresholds in the excitatory activity \y{et}, PV activity \y{pt}, and SOM activity \y{st}. These thresholds are minimum values beyond which the population activity can influence the postsynaptic population. Finally, the \y{pv\_opto} and \y{som\_opto} parameters set the level of optogenetic \textit{inactivation}, i.e., positive values of \y{pv\_opto} inactivation PVs, and likewise for \y{som\_opto}. In the paper I flip these numbers for readability.

Next up:
\begin{verbatim}
p q=1.3,tl=1
p taud1=150,taud2=10
p tde1=150,tde2=2
p lat=.65 
\end{verbatim}
\y{q} is the strength of the thalamic inputs into the model. \y{tl} is the decay time constant in the exponential function \y{dan}. The depression time constants are \y{taud1} and \y{taud2}, again in the same units, so $\tau_{D_1} = 1500$ms and $\tau_{D_2} = 100$ms. We can ignore \y{tde1,tde2} as they are not used in the code. The parameter \y{lat} determines the strength of lateral thalamic inputs.

The next chunk of code sets up the stimuli in the forward suppression paradigm.
\begin{verbatim}
par t0=10,dur=5,isi=2

t1on=t0
t1off=t1on+dur
t2on=t1off+isi
t2off=t2on+dur

t3on=t2off+isi
t3off=t3on+dur
t4on=t3off+isi
t4off=t4on+dur

t5on=t4off+isi
t5off=t5on+dur
t6on=t5off+isi
t6off=t6on+dur

t7on=t6off+isi
t7off=t7on+dur
t8on=t7off+isi
t8off=t8on+dur

# thalamic inputs with fast timescale depression
gt1(x)=heav(x-t1on)*heav(t1off-x)*dan(x-t1on)
gt2(x)=heav(x-t2on)*heav(t2off-x)*dan(x-t2on)
gt3(x)=heav(x-t3on)*heav(t3off-x)*dan(x-t3on)
gt4(x)=heav(x-t4on)*heav(t4off-x)*dan(x-t4on)

gt5(x)=heav(x-t5on)*heav(t5off-x)*dan(x-t5on)
gt6(x)=heav(x-t6on)*heav(t6off-x)*dan(x-t6on)
gt7(x)=heav(x-t7on)*heav(t7off-x)*dan(x-t7on)
gt8(x)=heav(x-t8on)*heav(t8off-x)*dan(x-t8on)

par mode=1
i1(x)=g1*q*if(mode-1)then(0)else(gt1(x))
i2(x)=g2*q*if(mode-2)then(gt2(x))else(gt1(x)+gt2(x))
i3(x)=g3*q*if(mode-3)then(0)else(gt1(x))
\end{verbatim}

The functions \y{gt1(x)} through \y{gt8(x)} represent 8 different stimuli with stimulus onset times of \y{t1on} through \y{t8on}, stimulus offset times of \y{t1off} through \y{t8off}. The \y{heav} functions are heaviside functions, and the multiplication of two of them as written forms a square pulse. This pulse is then modulated by the exponential decay function \y{dan}. For the forward suppression paradigm, we only need two stimuli, so we only use \y{gt1} and \y{gt2} for the rest of the code.

The \y{mode} parameter sets which unit receives the first input. If \y{mode=1}, then
\begin{verbatim}
i1(x)=g1*q*gt1(x)
i2(x)=g2*q*gt2(x)
i3(x)=0
\end{verbatim}
For more information on how the if statement works in XPP, see the bottom of \url{http://www.math.pitt.edu/~bard/xpp/help/xppodes.html}. If the \y{mode} parameter is set to \y{mode=2}, then
\begin{verbatim}
i1(x)=0
i2(x)=g2*q*(gt1(x)+gt2(x))
i3(x)=0
\end{verbatim}
and so on. The terms \y{g1} and \y{g2} are variables and satisfy the equations
\begin{verbatim}
g1'=(1-g1)/taud1-i1(y)/taud2
g2'=(1-g2)/taud1-i2(y)/taud2
g3'=(1-g3)/taud1-i3(y)/taud2
\end{verbatim}
where \y{y'=1} and all \y{g1,g2,g3} are initialized with a value of 1. The trivial ODE \y{y'=1} is needed for technical reasons related to XPP that isn't too important for now. Anyway, as \y{i1(y)} activates, it ``depletes'' vesicles and forces \y{g1} to decay on a fast timescale. With no other activation of \y{i1}, \y{g1} recovers slowly back to its original value of 1.

Now we finally reach the actual model equations:
\begin{verbatim}
u1'=-u1+f(wee*u1-(wep-damp*(1-g1))*p1-(wes+samp*(1-g1))*s1-et+i1(y)+i2(y)*lat+wee2*u2/1.5)
p1'=(-p1+f(wpe*u1-wpp*p1-wps*s1-pt +i1(y)+i2(y)*lat +wpe2*u2-pv_opto ))/tau1
s1'=(-s1+f(wse*u1-wsp*p1-wss*s1-st                  +wse2*u2-som_opto ))/tau2

u2'=-u2+f(wee*u2-(wep-damp*(1-g2))*p2-(wes+samp*(1-g2))*s2-et+(i1(y)+i3(y))*lat+i2(y)+wee2*(u1+u3)/2)
p2'=(-p2+f(wpe*u2-wpp*p2-wps*s2-pt +(i1(y)+i3(y))*lat+i2(y)+wpe2*(u1+u3)/2 -pv_opto ))/tau1
s2'=(-s2+f(wse*u2-wsp*p2-wss*s2-st                         +wse2*(u1+u3)/2 -som_opto ))/tau2

u3'=-u3+f(wee*u3-(wep-damp*(1-g3))*p3-(wes+samp*(1-g3))*s3-et+i2(y)*lat+i3(y)+wee2*u2/1.5)
p3'=(-p3+f(wpe*u3-wpp*p3-wps*s3-pt +i2(y)*lat+i3(y) +wpe2*u2-pv_opto ))/tau1
s3'=(-s3+f(wse*u3-wsp*p3-wss*s3-st                  +wse2*u2-som_opto ))/tau2
\end{verbatim}
These are the implemented versions of the 3 unit equations from the manuscript:
Th first or left unit satisfies
\begin{align*}
 \tau_u u_1' &= -u_1 + f( w_{ee} u_1 - (w_{ep}(t)-aD_1)p_1 - (w_{es}(t)+b F_1)s_1 + qI_1(t) + w_{ee}^* u_2),\\
 \tau_p p_1' &= -p_1 + f( w_{pe} u_1 - w_{pp}p_1 - w_{ps}s_1 + I_\text{Opt,PV} + qI_1(t) + w_{pe}^* u_2),\\
 \tau_s s_1' &= -s_1 + f( w_{se} u_1 - w_{sp}p_1 - w_{ss}s_1 + I_\text{Opt,SOM} + w_{se}^* u_2),
\end{align*}
where $I_1 = i_1(t) +i_2(t)\alpha$. The second, or center unit, satisfies
\begin{align*}
 \tau_u u_2' &= -u_2 + f( w_{ee} u_2 - (w_{ep}(t)-aD_2)p_2 - (w_{es}(t)+b F_2)s_2 + qI_2(t) + w_{ee}^* (u_1+u_3)/2),\\
 \tau_p p_2' &= -p_2 + f( w_{pe} u_2 - w_{pp}p_2 - w_{ps}s_2 + I_\text{Opt,PV} + qI_2(t) + w_{pe}^* (u_1+u_3)/2),\\
 \tau_s s_2' &= -s_2 + f( w_{se} u_2 - w_{sp}p_2 - w_{ss}s_2 + I_\text{Opt,SOM} + w_{se}^* u_2),
\end{align*}
where $I_2(t) = (i_1(t)+i_3(t))\alpha +i_2(t)$. Finally, the third, or right unit, satisfies
\begin{align*}
 \tau_u u_3' &= -u_3 + f( w_{ee} u_3 - (w_{ep}(t)-aD_3)p_3 - (w_{es}(t)+b F_3)s_3 + qI_3(t) + w_{ee}^* u_2),\\
 \tau_p p_3' &= -p_3 + f( w_{pe} u_3 - w_{pp}p_3 - w_{ps}s_3 + I_\text{Opt,PV} + qI_3(t) + w_{pe}^* u_2),\\
 \tau_s s_3' &= -s_3 + f( w_{se} u_3 - w_{sp}p_3 - w_{ss}s_3 + I_\text{Opt,SOM} + w_{se}^* u_2), 
\end{align*}
where $I_3 = i_2(t) +i_3(t)\alpha$.

The remaining lines are mostly related to using the XPPAUTO GUI. Choices that affect the integration are
\begin{verbatim}
@ bounds=100000
@ maxstor=1000000
@ total=200
@ dt=.1
\end{verbatim}
\y{bounds} is simply the maximum possible value of any solution before XPPAUTO gives up. \y{maxstor} sets the max storage XPPAUTO is allowed to take from memory. \y{total} is the total time of the simulation, and \y{dt} is the time step.

\subsection{Python Visualization}
These XPP ode files can be run using the XPP GUI, but we will run them using Python using \y{Py\_XPPCALL}. I have tried to include enough documentation to help users understand the basics \url{https://github.com/youngmp/Py_XPPCALL}. The Python files that run the XPP files are are located in \url{https://github.com/geffenlab/park_geffen/rate_models}. The Python file that reads the ODE file from the example above is called \y{cols3\_fs.py}. You should be able to run the python file directly.
\begin{verbatim}
$ python cols3_fs.py
\end{verbatim}
There will be lots of text that appears. These were mostly for debugging purposes and can be turned off by commenting the print statements in this file and in \y{xppcall.py}. After the script runs, a few plots should appear. I like to set up the plotting framework in the individual Python files before finalizing them in \y{generate\_figures.py}.

The \y{main} function calls all the relevant files. In brief, the loop runs three different stimulus patterns. The first stimulates the left unit then the center unit with 50ms tones and a 20ms inter-stimulus-interval. The second stimulates the center unit twice with the same time intervals. The third stimulates the right unit then the center unit with the same time intervals. This is a simple version of the forward suppression paradigm. These three sets of stimuli are repeated for PV inactivation and SOM inactivation. The max firing rate response is then recorded into a matrix and plotted.

Much of this logic appears in its final form in the function \y{r3\_s3\_fs\_full} in \y{generate\_figures.py}. The difference is that the function \y{r3\_s3\_fs\_full} includes the spiking model simulation and plots. I've tried to comment on most things... Let me know if you have any questions.

The remaining Python files for simulating the rate model, \y{cols3\_ssa} and \y{natan2015\_simple\_linear} have a similar construction. The \y{main} function in each script contains all the preliminary plotting/visualization code, which you can use to tinker and learn the ins and outs of the python/xpp interface. Seriously, play with those all you want.

The final version of the plotting functions appear in \y{generate\_figures.py}. Don't change those unless you're familiar with the code.


\section{Spiking Models}

\section{Generating Figures}

\y{generate\_figures.py} is the master plotting function for the paper. Most figure papers are generated here, except for a couple that were created in Inkscape. This file serves as a technical documentation of the code. In the main function, comments note which function belong to which figure.

\begin{thebibliography}{1}

\bibitem{ermentrout2002simulating}
Bard Ermentrout.
\newblock {\em Simulating, analyzing, and animating dynamical systems: a guide
  to XPPAUT for researchers and students}, volume~14.
\newblock Siam, 2002.

\bibitem{natan2015complementary}
Ryan~G Natan, John~J Briguglio, Laetitia Mwilambwe-Tshilobo, Sara~I Jones, Mark
  Aizenberg, Ethan~M Goldberg, and Maria~Neimark Geffen.
\newblock Complementary control of sensory adaptation by two types of cortical
  interneurons.
\newblock {\em Elife}, 4, 2015.

\bibitem{xppy}
Ilya Prokin and Youngmin Park.
\newblock Pyxppcall.
\newblock \url{https://github.com/iprokin/Py_XPPCALL}, 2017.

\bibitem{Stimberg595710}
Marcel Stimberg, Romain Brette, and Dan F.~M. Goodman.
\newblock Brian 2: an intuitive and efficient neural simulator.
\newblock {\em bioRxiv}, 2019.

\bibitem{tsodyks1997neural}
Misha~V Tsodyks and Henry Markram.
\newblock The neural code between neocortical pyramidal neurons depends on
  neurotransmitter release probability.
\newblock {\em Proceedings of the national academy of sciences},
  94(2):719--723, 1997.

\bibitem{yarden2017stimulus}
Tohar~S Yarden and Israel Nelken.
\newblock Stimulus-specific adaptation in a recurrent network model of primary
  auditory cortex.
\newblock {\em PLoS computational biology}, 13(3):e1005437, 2017.

\end{thebibliography}
\end{document}
